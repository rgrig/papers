\documentclass[preprint]{sigplanconf} % <<<

\usepackage{amsmath}
\usepackage{amssymb}
\usepackage{graphics}
\usepackage[latin1]{inputenc}
\usepackage{microtype}  % do not remove
\usepackage{pygmentize}
\usepackage{xcolor}
\usepackage[colorlinks]{hyperref}

\RecustomVerbatimEnvironment{Verbatim}{BVerbatim}{}
\definecolor{darkblue}{rgb}{0,0,0.4}
\definecolor{verylightgray}{rgb}{0.9,0.9,0.9}
% comment the next line for printing
\hypersetup{colorlinks,linkcolor=darkblue,citecolor=darkblue,urlcolor=darkblue}
\hypersetup{
  pdftitle={A Language for Specifying Safety Temporal Properties of Object-Oriented Programs},
  pdfauthor={Dino Distefano and Radu Grigore and Rasmus Lerchedahl Petersen}}

\titlebanner{DRAFT}
\title{A Language for Specifying Safety Temporal Properties of Object-Oriented Programs}
\authorinfo{Dino Distefano \and Radu Grigore \and Rasmus Lerchedahl Petersen}{Queen Mary, University of London}{{\rm\{}ddino,rgrig,rusmus{\rm\}}@eecs.qmul.ac.uk}

\newcommand{\N}{\ensuremath{\mathbb{N}}}
\newcommand{\pmap}{\rightharpoonup}
\newcommand{\set}[1]{\ensuremath{\mathsf{#1}}}

\overfullrule=5pt
\showboxdepth=10
\showboxbreadth=30
% >>>
\begin{document}
\maketitle

\begin{abstract} % <<<
TODO
\end{abstract}
\category{D.2.1}{Software Engineering}{Requirements/Specifications}
\terms Languages, Verification
\keywords Safety, Temporal Properties, Object-Oriented

% >>>
\section{Introduction} % <<<

Our goal is a tool that checks safety properties of large Java projects.
The properties should be given by the user; they should not be hard-coded.
This article presents a language of properties.
We tried to make it simple and intuitive, so that programmers would use it.

% >>>
\section{Example} % <<<

The program in~\autoref{fig:cme} fails with an exception.
It is illegal to modify the underlying collection while iterating.

\begin{figure}\centering
\include{Cme}
\caption{Throws \textit{ConcurrentModificationException}.}
\label{fig:cme}
\end{figure}

% >>>
\section{Syntax} % <<<

% >>>
\section{Semantics} % <<<

The input seen by an automaton is a sequence of events.
\begin{align}
\set{Event}&=\{\mathtt{call}, \mathtt{return}\}\times \set{MethodId} \times (\N \pmap \set{Value}) \\
\set{Input}&=\N\pmap\set{Event}
\end{align}

%>>>
\section{Testing} % <<<

% >>>
\section{Future Work} %<<<

%>>>
\section{Related Work} %>>>

%>>>
\section{Conclusions} %<<<

%>>>

\end{document}
% vim:spell errorformat=%f\:%l-%m,%f\:%l\:%m,%f\:%m
% vim:fmr=<<<,>>>:

