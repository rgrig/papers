\def\pmap{\rightharpoonup}
\centerline{\bf mop}

\medskip\noindent This note summarizes the article {\it Semantics and Algorithmics for Parametric Monitoring\/}.
\medskip

The model of computation is a set of automaton instances running in parallel.
The automata are deterministic and may have an infinite number of states.
Each instance sees only a slice of the trace.
There is a set~$X$ of {\it parameters\/} and a set~$V$ of {\it values\/}; there is one automaton instance running for each binding $\theta\in X\pmap V$.
Each event has a name $e\in E$ and a binding $\theta'\in X\pmap V$.
The automaton instance for~$\theta$ sees exactly the events carrying bindings $\theta'\le\theta$.
The order is the usual order on partial maps:
$\theta'\le\theta$ means that the two maps agree on the domain of~$\theta'$, which is included in the domain of~$\theta$.

An automaton has a possibly infinite set~$S$ of states and a transition function $\delta\in(S\times E)\to S$.
Notice that the transition function sees only the name of the event, but not the bindings.
The accepting states of instance~$\theta$ are those in the set $\alpha(\theta)$, where $\alpha\in(X\pmap V)\to S\to 2$.
The automaton is essentially described by the pair~$(\alpha,\delta)$.

The paper culminates with an online algorithm for slicing traces.
Its simplest version is straightforward, but not simple.
To get the intuition behind it, please consider the following problem.
Suppose we have a hash function for sequences of integers that may be computed online (one integer at a time), such as $h({\bf y}z)=31h({\bf y})+z$.
We are given a sequence~${\bf y}$ of integers and we are asked to process it one integer at a time and construct a data structure that can answer queries of the form:
`What is the hash of the subsequence of integers $\le z$?''

\medskip

Consider the sequence ${\bf y}=[5,9,5,9,4]$.
The query `$\le5$' should be answered by the hash of $[5,5,4]$;
the query `$\le4$' should be answered by the hash of $[4]$;
the query `$\le0$' should be answered by the hash of $[]$.
The are $3$~distinct values from the input sequence partition the integers into $3+1$~intervals.
$$Z=(-\infty.\,.\,4)\cup[4.\,.\,5)\cup[5.\,.\,9)\cup[9.\,.\,\infty)$$
Each interval corresponds to one possible answer.
For example, all queries `$\le z$' with $z\in[4.\,.\,5)$ have the same answer as the query~`$\le4$'.
So, the preprocessing phase could build the dictionary
$$d=\{\,
-\infty\mapsto h([]),\;
4\mapsto h([4]),\;
5\mapsto h([5,5,4]),\;
9\mapsto h([5,9,5,9,4])
\,\}$$
Answering a query amounts to a predecessor search in~$d$, in $O(\lg\lg|d|)$~time.

The straightforward solution is to start with the dictionary $\{\,-\infty\mapsto h([])\,\}$, which corresponds to the empty sequence, and update it for each integer in the input sequence.
For example, once the sequence $[5,9,5,9]$ was processed the state would be the dictionary
$$d'=\{\,
-\infty\mapsto h([]),\;
5\mapsto h([5,5]),\;
9\mapsto h([5,9,5,9])
\,\}$$
Processing the last integer, which is~$4$, amounts to updating $d'$ into~$d$.

The preprocessing algorithm is the following.
$$\vbox{
  \settabs\+\quad&\quad&\cr
  \+    $d[-\infty]:=0$\cr
  \+    for each $y$ in ${\bf y}$, in order\cr
  \+&     $x:=\max\{x'\in{\it keys}(d):x'\le y\}$\cr
  \+&     $d[y]:=31d[x]+y$\cr
  \+&     for each $z$ in $\{z'\in{\it keys}(d):y<z'\}$\cr
  \+&&      $d[z]:=31d[z]+y$\cr
}$$

Why is this algorithm correct?
The {\it slice\/}~${\bf y}/z$ is the subsequence of~${\bf y}$ that retains integers~$\le z$.
Clearly, $({\bf x}\cdot{\bf y})/z=({\bf x}/z)\cdot({\bf y}/z)$, where $\cdot$ denotes sequence concatenation.
In particular $({\bf y}\cdot[y])/z$ is $({\bf y}/z)\cdot([y]/z)$;
and $[y]/z$ is $[y]$~or~$[]$, depending on whether $y\le z$.
Intuitively, this means that slices can be computed incrementally, which in turn implies that the dictionary~$d$ can be computed incrementally.
The following are invariants of the outer loop:
$$\eqalign{
  {\it keys}(d)&=\{ \max{\bf x} : \hbox{${\bf x}={\bf y}/z$ for some integer $z$} \}\cr
  d[z]&=h({\bf y}/z)\quad\hbox{for all $z\in{\it keys}(d)$}\cr
}$$

Is this algorithm fast?
It takes $\sim mn$~time, where $m$~is the length of the sequence and $n$~is the number of distinct integers in the sequence.
Still, the running time is cut roughly in half if one can quickly
(1)~find the (immediate) predecessor~$x$ of the integer~$y$ being processed, and
(2)~iterate over all~$z$ that are strictly bigger than~$y$.
On average, it's reasonable to expect about half of the keys of~$d$ to be bigger than the current~integer.

\medskip

Let's now go back to the original problem, that of slicing traces of events.
It is essentially the same:
The `hash' of ${\bf y}$ is the state of an automaton instance after procesing~${\bf y}$.
The {\it main difference\/} is that the elements of the sequence, which are parameter bindings $\theta:X\pmap V$, are not equipped with a total order, only a partial order.
In general in a partial order there is no such thing as an immediate predecessor.
However, partial maps $\theta_1$~and~$\theta_2$ can be joined into $\theta_1\cup\theta_2$ if they agree on their common domain.
It turns out that
$${\it keys}(d)=\{\cup{\bf x} : \hbox{${\bf x}={\bf y}/\theta$ for some $\theta\in X\pmap V$}\}$$
is closed under~$\cup$, which is why it contains a unique immediate predecessor for any parameter binding.
The search for this intermediate predecessor isn't smart: starting with $\theta$, mop first sees if $\theta$ is in~${\it keys}(d)$, then tries to drop $1$~binding from~$\theta$, then $2$~bindings, and so on.
The main thing mop does to speed up computation is that it maintains for each element of ${\it keys}(d)$ the set of strictly bigger elements, so that the inner loop of the algorithm is sped up.


\bye
% vim:wrap:linebreak:fmr=<<<,>>>:nosi:spell:
